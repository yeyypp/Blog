% !TEX TS-program = xelatex
% !TEX encoding = UTF-8 Unicode
% !Mode:: "TeX:UTF-8"

\documentclass{resume}
\usepackage{zh_CN-Adobefonts_external} % Simplified Chinese Support using external fonts (./fonts/zh_CN-Adobe/)
%\usepackage{zh_CN-Adobefonts_internal} % Simplified Chinese Support using system fonts
\usepackage{linespacing_fix} % disable extra space before next section
\usepackage{cite}

\begin{document}
\pagenumbering{gobble} % suppress displaying page number

\name{叶帅}

\basicInfo{
  \email{shuaiye_2019@163.com} \textperiodcentered\ 
  \phone{(+86) 186-1018-8361} \textperiodcentered\ 
  \github[https://github.com/yeyypp]{https://github.com/yeyypp}
  }
 
\section{教育背景}
\datedsubsection{\textbf{纽卡斯尔大学}, 英国}{2017 - 2018}
\textit{硕士}\ 计算机科学
\datedsubsection{\textbf{四川大学}, 四川,成都}{2012- 2016}
\textit{学士}\ 工程力学

\section{个人项目}
\datedsubsection{\textbf{Min-ShoppingMall}}
\basicInfo{迷你商城, 项目地址:https://github.com/yeyypp/Mini-ShoppingMall}
\begin{itemize}
  \item Spring-Security实现鉴证,授权
  \item Spring-Jpa实现数据库交互
  \item 
\end{itemize}

\datedsubsection{\textbf{Bear-Map}}
\basicInfo{Bear-Map, 项目地址:https://github.com/yeyypp/skeleton-sp18/tree/master/proj3}
\begin{itemize}
  \item 根据用户输入计算出相应地图信息
  \item 实现Graph结构存储相关地图信息
  \item 实现Trie结构,完成自动补全及搜索
  \item 实现Dijkstra,A*算法完成寻路
\end{itemize}



% Reference Test
%\datedsubsection{\textbf{Paper Title\cite{zaharia2012resilient}}}{May. 2015}
%An xxx optimized for xxx\cite{verma2015large}
%\begin{itemize}
%  \item main contribution
%\end{itemize}

\section{技能}
% increase linespacing [parsep=0.5ex]
\begin{itemize}[parsep=0.5ex]
  \item 熟悉Java(集合,多线程,NIO等),了解常用数据结构及算法
  \item 了解Spring-Boot,有相关组件使用经验(Security,Jpa,Cloud)
  \item 熟悉常用开发工具(IDEA, Git, Github)
\end{itemize}


\section{其他}
% increase linespacing [parsep=0.5ex]
\begin{itemize}[parsep=0.5ex]
  \item GitHub: https://github.com/yeyypp
  \item 学习笔记:https://github.com/yeyypp/Blog
  \item 语言: 英语 - 良好(IELTS 6.5)
\end{itemize}

%% Reference
%\newpage
%\bibliographystyle{IEEETran}
%\bibliography{mycite}
\end{document}
