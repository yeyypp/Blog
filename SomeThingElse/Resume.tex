% !TEX TS-program = xelatex
% !TEX encoding = UTF-8 Unicode
% !Mode:: "TeX:UTF-8"

\documentclass{resume}
\usepackage{zh_CN-Adobefonts_external} % Simplified Chinese Support using external fonts (./fonts/zh_CN-Adobe/)
%\usepackage{zh_CN-Adobefonts_internal} % Simplified Chinese Support using system fonts
\usepackage{linespacing_fix} % disable extra space before next section
\usepackage{cite}

\begin{document}
\pagenumbering{gobble} % suppress displaying page number

\name{叶帅}

\basicInfo{
  \email{18610188361@163.com} \textperiodcentered\
  \phone{(+86) 186-1018-8361} \textperiodcentered\
  \github[https://github.com/yeyypp]{https://github.com/yeyypp} \textperiodcentered\
  }

\section{教育经历}
\datedsubsection{\textbf{纽卡斯尔大学}, 英国}{2017.9 -- 2019.1}
    \textbf{硕士} ,计算机科学}
\datedsubsection{\textbf{四川大学}, 成都}{2012 -- 2016}
    \textbf{学士} ,工程力学}

\section{个人项目}
\datedsubsection{\textbf{Bear Map}}{\url{https://github.com/yeyypp/skeleton-sp18/tree/master/proj3}}
\begin{itemize}
    \item 设计,实现后端模块部分功能
    \item Rasterer:根据输入的坐标及其他参数,匹配相应的地图图片
    \item GraphDB:实现图数据结构,存储相应地图信息
    \item GraphBuildingHandler: 根据解析出的地图数据,添加相应信息到GraphDB
    \item Router:实现Dijkstra, A*寻路算法
    \item Autocomplete:设计Trie数据结构,实现搜索自动补全功能
\end{itemize}

% Reference Test
%\datedsubsection{\textbf{Paper Title\cite{zaharia2012resilient}}}{May. 2015}
%An xxx optimized for xxx\cite{verma2015large}
%\begin{itemize}
%  \item main contribution
%\end{itemize}

\section{技能}
% increase linespacing [parsep=0.5ex]
\begin{itemize}[parsep=0.5ex]
  \item 编程语言: 熟悉Java基础,理解IO,集合,多线程等内容,对JVM有一定了解。
  \item 基础知识:熟悉常用数据结构及算法,了解TCP/IP网络,操作系统相关知识
  \item 框架:熟悉SpringBoot,Mybatis
  \item 数据库:熟悉MySQL,Redis
  \item 平台: Windows,Linux
  \item 开发工具:IDEA,Maven,Vim,Git
\end{itemize}



\section{其他}
% increase linespacing [parsep=0.5ex]
\begin{itemize}[parsep=0.5ex]
  \item 语言: 英语 - 良好(IELTS 6.5)
\end{itemize}

%% Reference
%\newpage
%\bibliographystyle{IEEETran}
%\bibliography{mycite}
\end{document}